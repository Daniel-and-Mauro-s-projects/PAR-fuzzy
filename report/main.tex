\documentclass[11pt]{article}
\usepackage{hyperref}
\usepackage{graphicx}
\usepackage{caption}
\usepackage{geometry}
\usepackage{placeins}
\usepackage{enumitem}
\usepackage{multirow}
\usepackage{float}
\usepackage{wrapfig}
\usepackage{url}
\usepackage{amsmath}
\usepackage{algorithm}
\usepackage{biblatex}
\usepackage{algpseudocode}
\usepackage{tikz}
\usepackage{schemata}
\usepackage{tabularx}
\usepackage{makecell}
\usepackage[toc,page]{appendix}

\addbibresource{bibliography.bib}


% Set page margins
\geometry{a4paper, margin=1.5cm}

% Set paragraph and spacing
\setlength{\parindent}{0em} % No indentation (annoying)
\setlength{\parskip}{0.5em} % Small space between paragraphs

\graphicspath{{../figures}}

\begin{document}

% TODO: update title
\begin{titlepage}
    \centering
    \vspace*{2cm}
    
    
    {\Huge\bfseries Fuzzy Expert System to Detect \\ Phishing in Websites\par}
    \vspace{1cm}
    % {\large A Comparative Analysis of k-Nearest Neighbors and SVM Classifiers\par}
    
    \vspace{2cm}
    
    {\large
    Dániel MÁCSAI \\ 
    Ismael RUIZ GARCIA \\ 
    Mauro VÁZQUEZ CHAS
    \par}
    
    \vspace{2cm}
    
    {\large
    \textbf{Master in Artificial Intelligence}
    \par}
    
    \includegraphics[width=0.4\textwidth]{Logo_URV.png}\par\vspace{1cm}

    \vspace{1cm}

    {\large
    Planning and Approximate Reasoning\\
    Delivery 3
    \par}
    
    \vspace{1cm}
    
    {\large\bfseries 15th December 2024\par}
    
\end{titlepage}


% Index
\newpage

\tableofcontents
\newpage

%---------------------------------------------------------------------------------------------------------------------------------
\section{Introduction}
For this work, we 
% TODO  write introduction

\section{Task 1}
To design the fuzzy expert system to detect phising websites, we consulted \cite{main_paper}. In this paper, they list 87 posible features (boolean, floats and integers) that could matter in the detection of phising websites. 

\subsection{Chosen Features}
Chosen variables

We selected features from each of the three categories presented in the paper: URL-based features, content features and external features. 

URL-based features:
\begin{itemize}
    \item Redirections (38) (int)
    URL redirection is a technique used to open pages with different URLs
than those initially selected by users. This is useful to prevent access to
broken links when web pages are moved. URLs can be redirected to pages
with the same domain (i.e. internal redirection) or to pages from different
domains (i.e., external redirections). However, redirection can also be used
for hostile purposes.
    \item Domain Age (83) (int)
\end{itemize}

Content features:
\begin{itemize}
    \item Ratio external hyperlinks (59) (float)
\end{itemize}

External features:
\begin{itemize}
    \item Google Index (86) (boolean)
    \item Page rank (87) (int)
\end{itemize}

\subsection{Fuzzy Sets}
For the output variable, you must define four fuzzy categories: safe, weakly 
suspicious, strongly suspicions and phishing. In addition to the category class, we want to get a 
numerical score in the range of 0..100


\subsubsection{Abnormal Subdomains}





% Bibliography
\newpage
\printbibliography


\end{document}
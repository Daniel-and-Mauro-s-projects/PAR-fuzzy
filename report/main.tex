\documentclass[11pt]{article}
\usepackage{hyperref}
\usepackage{graphicx}
\usepackage{caption}
\usepackage{geometry}
\usepackage{placeins}
\usepackage{enumitem}
\usepackage{multirow}
\usepackage{float}
\usepackage{wrapfig}
\usepackage{url}
\usepackage{amsmath}
\usepackage{algorithm}
\usepackage{biblatex}
\usepackage{algpseudocode}
\usepackage{tikz}
\usepackage{schemata}
\usepackage{tabularx}
\usepackage{makecell}
\usepackage[toc,page]{appendix}

\addbibresource{bibliography.bib}


% Set page margins
\geometry{a4paper, margin=1.5cm}

% Set paragraph and spacing
\setlength{\parindent}{0em} % No indentation (annoying)
\setlength{\parskip}{0.5em} % Small space between paragraphs

\graphicspath{{../figures}}

\begin{document}

% TODO: update title
\begin{titlepage}
    \centering
    \vspace*{2cm}
    
    
    {\Huge\bfseries Fuzzy Expert System to Detect \\ Phishing in Websites\par}
    \vspace{1cm}
    % {\large A Comparative Analysis of k-Nearest Neighbors and SVM Classifiers\par}
    
    \vspace{2cm}
    
    {\large
    Dániel MÁCSAI \\ 
    Ismael RUIZ GARCIA \\ 
    Mauro VÁZQUEZ CHAS
    \par}
    
    \vspace{2cm}
    
    {\large
    \textbf{Master in Artificial Intelligence}
    \par}
    
    \includegraphics[width=0.4\textwidth]{Logo_URV.png}\par\vspace{1cm}

    \vspace{1cm}

    {\large
    Planning and Approximate Reasoning\\
    Delivery 3
    \par}
    
    \vspace{1cm}
    
    {\large\bfseries 15th December 2024\par}
    
\end{titlepage}


% Index
\newpage

\tableofcontents
\newpage

%---------------------------------------------------------------------------------------------------------------------------------
\section{Introduction}
For this work, we 
% TODO  write introduction

\section{Task 1}
To design the fuzzy expert system to detect phishing websites, we consulted \cite{main_paper}. In this paper, they list 87 posible features (boolean, floats and integers) that could matter in the detection of phishing websites. 

\subsection{Chosen Features}
Chosen variables

We selected features from each of the three categories presented in the paper: URL-based features, content features and external features. 

URL-based features:
\begin{itemize}
    \item Phish Hints (51) (int)
    Phishing URLs use sensitive words to gain trust on visited web pages. The
number of such words in URLs is considered as phishing indicato
    \item Domain Age (83) (int)
\end{itemize}

Content features:
\begin{itemize}
    \item Ratio external hyperlinks (59) (float)
\end{itemize}

External features:
\begin{itemize}
    \item Google Index (86) (boolean) Phishing websites live for short times and are often accessible through direct links sent to users in emails, they do not need to be indexed by
    Google. Web pages not indexed by Google are supposed phishing
    \item Page rank (87) (int) Phishing web pages are not very popular, hence, they suppose to save low page ranks compared with legitimate web pages. We use Openpagerank to get the value of this feature \textbf{In the end we chose Google Pagerank that is 0-10}
    
\end{itemize}

\subsection{Fuzzy Sets}
For the output variable, you must define four fuzzy categories: safe, weakly 
suspicious, strongly suspicions and phishing. In addition to the category class, we want to get a 
numerical score in the range of 0..100


\subsubsection{Abnormal Subdomains}





% Bibliography
\newpage
\printbibliography


\end{document}